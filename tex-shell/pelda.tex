
\documentclass[12pt,a4paper]{report}

\usepackage[T1]{fontenc}
\usepackage[magyar]{babel}

\usepackage{graphicx}
\usepackage{amssymb}
\usepackage{amsmath}
\usepackage{amsfonts}
\usepackage{amsthm}

\usepackage[margin=2.5cm,bindingoffset=1cm]{geometry}
\usepackage{lscape}    %külön elfordítható oldalakhoz
\usepackage[onehalfspacing]{setspace} %\singlespacing, \doublespacing

\usepackage{apacite}

\usepackage{booktabs}
\usepackage{tabularx}


%függvényrajzolgatásra (ezt én használtam, de neked nem biztos, hogy kell)
\usepackage{tikz}
\usetikzlibrary{shapes,arrows,mindmap,trees}
\usepackage{pgfplots}
\pgfplotsset{compat=1.12}

\usepackage{url}

\usepackage{fontspec}
\usepackage{unicode-math}
\setmainfont{Calibri}
\setmathfont{Asana-Math}  % a Calibri-hez illő matek betűtípus
                          % innen szedd le a .OTF fájlt: https://www.ctan.org/tex-archive/fonts/Asana-Math/?lang=en
                          % és telepítsd fel, ez után elvileg működni fog nálad is

\usepackage{microtype} % a betűk harmonikusabb elrendezése

\begin{document}

%% Fedőlap
\thispagestyle{empty}

\begin{center}
\LARGE
Eötvös Loránd Tudományegyetem \\
Társadalomtudományi Kar \\
ALAPKÉPZÉS

\vspace{60mm}

\Huge{CÍM}
\LARGE
\vspace{60mm}

%%%%%%%%%%%%%%%%%%%%%%%%%%%%%%%%%%%%%%%%%%%%%%%%%%%%%%%%%%%%%%%%%%%%%%%%%%%%%%%%%%%%%%%%%%%%%%%%%%%%%%%%%%%%%%%%%%%%%%%%%
\begin{tabular*}{\linewidth}{@{\extracolsep{\fill}}lr}
  Konzulens: & Készítette: \\
  xy & xy \\
   & NEPTUN kódod \\
   & alkalmazott közgazdaságtan szak
\end{tabular*}
%\vspace{20mm}
\vskip0pt plus1filll
2016. április
\end{center}
%%%%%%%%%%%%%%%%%%%%%%%%%%%%%%%%%%%%%%%%%%%%%%%%%%%%%%%%%%%%%%%%%%%%%%%%%%%%%%%%%%%%%%%%%%%%%%%%%%%%%%%%%%%%%%%%%%%%%%%%%


\newpage
\setcounter{page}{1} \pagenumbering{roman}

%% Ide írd a kivonatot
\begin{abstract}

Ide jön az absztrakt.

\end{abstract}

%% Tartalomjegyzék, ábrák és táblázatok jegyzéke
\microtypesetup{protrusion=false} % oldalszámok szebb illeszkedéséhez
\tableofcontents
\listoftables
\listoffigures
\microtypesetup{protrusion=true}

% Most hogy szépen megvan minden, már csak írni kell
% A \chapter{} új fejezetet definiál, ezen belül csinálhatsz \section{} és \subsection{} részeket (többek között).
% A \label{xy} nagyon hasznos, a szövegben \ref{xy} módon hivatkozhatsz az elem számára (1. fejezet pl),
% a \pageref{xy} az elem oldalszámát adja vissza.
%%%%%%%%%%%%%%%%%%%%%%%%%%%%%%%%%%%%%%%%%%%%%%%%%%%%%%%%%%%%%%%%%%%%%%%%%%%%%%%%%%%%%%%%%%%%%%%%%%%%%%%%%%%%%%%%%%%%%%%%%

\chapter{Egyik} \label{ch:bev}
\setcounter{page}{1} \pagenumbering{arabic}
%A dolgozat bevezetőjében mindenképpen szerepelnie kell egy konkrét
%kérdésnek, amire a hallgató a dolgozatban választ keres, illetve a témakör megjelölésének,
%amellyel kapcsolatos szakirodalmi áttekintést készít. A tartalmi kifejtés során fontos a logikus
%felépítés, a szakkifejezések megfelelő használata, ha szükséges azok tisztázása. Az összegező
%befejezésnek tartalmaznia kell egy konklúziót a bevezető kérdéssel kapcsolatban. 

Már \citeA{wooldridge} is megmondta, hogy \dots

Lorem ipsum dolor sit amet, consectetur adipiscing elit. Curabitur eu blandit metus. Cum sociis natoque penatibus et magnis dis parturient montes, nascetur ridiculus mus. Praesent pellentesque suscipit tempor. Integer vel quam quis tortor tempor aliquam. Vivamus porttitor quam eu erat pharetra, vel sodales mi accumsan. Nullam non hendrerit nibh. Integer a enim feugiat, ullamcorper mi eget, ultrices tortor. Fusce eu orci in dolor vulputate efficitur at quis metus. Nullam at turpis augue. Phasellus sed neque mauris. Fusce gravida pulvinar pharetra. In quis nulla vel lacus interdum maximus. Fusce cursus finibus mauris, sit amet porta nulla fermentum in. Quisque vitae magna quis urna fermentum ornare. Sed iaculis nec ligula vitae auctor. Vivamus posuere sem nec ante congue lacinia.

Etiam fringilla neque nec diam commodo aliquam. In euismod ligula at neque pretium, sed pellentesque ligula tincidunt. Donec molestie dictum tortor et vulputate. Vivamus pulvinar aliquet libero, et laoreet lectus luctus vitae. Integer porttitor justo magna, sit amet aliquam eros faucibus at. In tincidunt lorem nisl, fringilla lobortis ex dictum id. Vivamus gravida nunc sed porttitor malesuada. Sed vulputate est sit amet mattis hendrerit.

\chapter{Másik}

Suspendisse nunc purus, semper ut semper id, pretium quis quam. Proin blandit leo eu dui sollicitudin, et posuere ante pulvinar. Proin finibus ante eu congue scelerisque. Nam pharetra maximus dui et egestas. Donec pellentesque efficitur sapien, sed ultrices nunc consequat non. Curabitur tincidunt facilisis quam, a sodales risus vulputate vel. Morbi sollicitudin urna fermentum, volutpat massa non, convallis enim.

\section{Harmadik}

Suspendisse potenti. Fusce blandit semper ante, eu mollis leo viverra eu. Quisque ornare tortor ex, sit amet tristique nisi molestie ac. Cum sociis natoque penatibus et magnis dis parturient montes, nascetur ridiculus mus. Nulla eleifend dictum vulputate. Etiam vulputate dolor ut neque congue, quis tempus elit varius. Proin rutrum tortor porttitor elit pretium egestas. Sed eget dignissim urna, a eleifend nibh. Nullam id varius neque. Morbi a iaculis leo.











%%%%%%%%%%%%%%%%%%%%%%%%%%%%%%%%%%%%%%%%%%%%%%%%%%%%%%%%%%%%%%%%%%%%%%%%%%%%%%%%%%%%%%%%%%%%%%%%%%%%%%%%%%%%%%%%%%%%%%%%%

%% Ez helyezi el a hivatkozásjegyzéket
% Zotero-t használóknak: bibtex export segítségével lehet építgetni a saját .bib fájlunkat
\bibliographystyle{apacite}
\bibliography{szakdolg}     % ez a .tex fájllal egy mappában lévő szakdolg.bib nevű fájlt használja fel 
                            % a hivatkozásjegyzék item-jeinek forrásaként

% Hivatkozási parancsok a .bib fájlban szereplő elnevezések alapján, és azok megjelenési formája a szövegben:
% \cite{pete_2013} : (Pete, 2013)         \cite[3.~o.]{pete_2013} : (Pete, 2013, 3. o.)
% \citeA{pete_2013} : Pete (2013)         \citeA[3.~o.]{pete_2013} : Pete (2013, 3. o.)
% \citeyear{pete_2013} : (2013)
% és még mások, amiket nem ismerek...

%% Függelék: innentől kezdve a chapterek függelékként vannak jelölve a szövegben (A. függelék, B. függelék ...)
\appendix

\chapter{Valami}


Nullam sapien urna, egestas et gravida id, dictum id magna. Nam iaculis facilisis euismod. Integer semper vehicula quam id tempor. Duis non orci vel augue faucibus blandit eu in arcu. In cursus lacus est. Morbi sagittis ex sagittis, vehicula tortor vel, scelerisque mauris. Donec quis ullamcorper nunc. In non consectetur tellus, ut lobortis lorem. Suspendisse id tempus turpis.


\end{document}