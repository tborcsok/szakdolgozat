\documentclass[12pt,a4paper]{report}

\usepackage[T1]{fontenc}
\usepackage[magyar]{babel}

\usepackage{graphicx}
\usepackage{amssymb}
\usepackage[leqno]{amsmath}
\usepackage{amsfonts}
\usepackage{amsthm}
\usepackage[margin=2.5cm,bindingoffset=1cm]{geometry}
\usepackage{lscape}    %külön elfordítható oldalakhoz
\usepackage[onehalfspacing]{setspace} %\singlespacing, \doublespacing

\usepackage{apacite}
\usepackage{booktabs}
\usepackage{tabularx}


%függvényrajzolgatásra (ezt én használtam, de neked nem biztos, hogy kell)
\usepackage{tikz}
\usetikzlibrary{shapes,arrows,mindmap,trees}
\usepackage{pgfplots}
\pgfplotsset{compat=1.12}


\usepackage{url}

\usepackage{fontspec}
\usepackage{unicode-math}
\setmainfont{Calibri}
\setmathfont{Asana-Math}  % egy esztétikus, és a Calibri-hez illő matek betűtípus
                          % innen szedd le a .OTF fájlt: https://www.ctan.org/tex-archive/fonts/Asana-Math/?lang=en
                          % és telepítsd fel

\usepackage{microtype}


\begin{document}


%% Fedőlap
\thispagestyle{empty}

\begin{center}
\LARGE
Eötvös Loránd Tudományegyetem \\
Társadalomtudományi Kar \\
ALAPKÉPZÉS

\vspace{60mm}

\Huge{CÍM}
\LARGE
\vspace{60mm}


\begin{tabular*}{\linewidth}{@{\extracolsep{\fill}}lr}
  Konzulens: & Készítette: \\
  xy & xy \\
   & NEPTUN kódod \\
   & alkalmazott közgazdaságtan szak
\end{tabular*}


%\vspace{20mm}
\vskip0pt plus1filll
2016. április
\end{center}

\newpage
\setcounter{page}{1} \pagenumbering{roman}

%% Ide írd a kivonatot - ha nem terveztél ilyet, akkor tervezzél
\begin{abstract}

Ide jön az absztrakt.

\end{abstract}

%% Tartalomjegyzék, ábrák és táblázatok jegyzéke
\microtypesetup{protrusion=false} % oldalszámok szebb illeszkedéséhez
\tableofcontents
\listoftables
\listoffigures
\microtypesetup{protrusion=true}

% Most hogy szépen megvan minden, már csak írni kell
%%%%%%%%%%%%%%%%%%%%%%%%%%%%%%%%%%%%%%%%%%%%%%%%%%%%%%%%%%%%%%%%%%%%%%%%%%%%%%%%%%%%%%%%%%%%%%%%%%%%%%%%%%%%%%%%%%%%%%%%%%%%%%%%%%%%%%%%%%%%%%%%%%%%%%%%%%%%%%%%%%%%%%%%%%%%%%%%%%%%%%%%%%%%%%%%%%%%%%%%%%%%%%%%%%%%

\chapter{Bevezetés} \label{ch:bev}
\setcounter{page}{1} \pagenumbering{arabic}
%A dolgozat bevezetőjében mindenképpen szerepelnie kell egy konkrét
%kérdésnek, amire a hallgató a dolgozatban választ keres, illetve a témakör megjelölésének,
%amellyel kapcsolatos szakirodalmi áttekintést készít. A tartalmi kifejtés során fontos a logikus
%felépítés, a szakkifejezések megfelelő használata, ha szükséges azok tisztázása. Az összegező
%befejezésnek tartalmaznia kell egy konklúziót a bevezető kérdéssel kapcsolatban. 

\chapter{...}


%%%%%%%%%%%%%%%%%%%%%%%%%%%%%%%%%%%%%%%%%%%%%%%%%%%%%%%%%%%%%%%%%%%%%%%%%%%%%%%%%%%%%%%%%%%%%%%%%%%%%%%%%%%%%%%%%%%%%%%%%%%%%%%%%%%%%%%%%%%%%%%%%%%%%%%%%%%%%%%%%%%%%%%%%%%%%%%%%%%%%%%%%%%%%%%%%%%%%%%%%%%%%%%%%%%%


%% Ez helyezi el a hivatkozásjegyzéket
% Zotero-t használóknak: bibtex export segítségével lehet építgetni a saját .bib fájlunkat
\bibliographystyle{apacite}
\bibliography{szakdolg} %ez a .tex fájllal egy mappában lévő szakdolg.bib nevű fájl használja fel a hivatkozásjegyzék item-jeinek forrásaként

%% Függelék: innentől kezdve a chapterek függelékként vannak jelölve a szövegben (A. függelék, B. függelék ...)
\appendix

\chapter{valami}


\end{document}



