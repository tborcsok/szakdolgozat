\documentclass[12pt,a4paper]{article}
\usepackage[magyar]{babel}
\usepackage[T1]{fontenc}

\author{<Neved>}
\title{<A szakdolgozatod címe>}

\usepackage{graphicx}
\usepackage{amssymb}
\usepackage{amsmath}
\usepackage{amsfonts}
\usepackage{amsthm}
\usepackage{mathtools}
\allowdisplaybreaks

\usepackage[margin=2.5cm,bindingoffset=1cm]{geometry}
\usepackage{lscape}    %külön elfordítható oldalakhoz
\linespread{1.33} % másfeles sorköz, lásd: https://www.sharelatex.com/learn/Paragraph_formatting#/Line_spacing

\usepackage{apacite}    % a hivatkozásokat kezelő csomag
                        % leírás: http://ftp.cvut.cz/tex-archive/biblio/bibtex/contrib/apacite/apacite.pdf
% a hivatkozásjegyzéket magyarra fordító rész:
\renewcommand{\BRetrieved}[1]{Megtekintés~időpontja:~{#1}.~Hozzáférés:\ }% Websites; note the space.
\renewcommand{\BRetrievedFrom}{Hozzáférés:\ }% Websites; note the space.
\renewcommand{\BED}{Szerk.\hbox{}}%          % editor
\renewcommand{\BEDS}{Szerk.\hbox{}}%        % editors
\renewcommand{\BBAA}{és}%  between authors in parenthetical cites and ref. list
\renewcommand{\BBAB}{és}% between authors in in-text citation
\renewcommand{\bibnodate}{n.d.\hbox{}}%   % ``no date''

\addto{\captionsmagyar}{\renewcommand{\abstractname}{Absztrakt}}

\usepackage{booktabs}
\usepackage{tabularx}


\usepackage{hyperref}
\hypersetup{
	pdfstartview={XYZ null null 1.00},
	bookmarksopen=true,
    colorlinks,
    citecolor=black,
    filecolor=black,
    linkcolor=black,
    urlcolor=blue,
    unicode}

%Calibri betűtípus beállítása
\usepackage{fontspec}
\usepackage{unicode-math}
\setmainfont{Calibri}
\setmathfont{Asana-Math}  % a Calibri-hez illő matek betűtípus
                          % innen szedd le a .OTF fájlt: https://www.ctan.org/tex-archive/fonts/Asana-Math/?lang=en
                          % és telepítsd fel, ez után elvileg működni fog nálad is

\usepackage{microtype} % a betűk harmonikusabb elrendezése

\begin{document}

%% Fedőlap
\thispagestyle{empty}

\begin{center}
\LARGE
Eötvös Loránd Tudományegyetem \\
Társadalomtudományi Kar \\
ALAPKÉPZÉS

\vspace{60mm}

\makeatletter
\Huge{\@title}
\makeatother
\LARGE
\vspace{60mm}

%%%%%%%%%%%%%%%%%%%%%%%%%%%%%%%%%%%%%%%%%%%%%%%%%%%%%%%%%%%%%%%%%%%%%%%%%%%%%%%%%%%%%%%%%%%%%%%%%%%%%%%%%%%%%%%%%%%%%%%%%
\makeatletter

\begin{tabular*}{\linewidth}{@{\extracolsep{\fill}}lr}
  Konzulens: & Készítette: \\
  konzulensem neve & \@author \\    
   & NEPTUN kódod \\
   & <> szak
\end{tabular*}
%\vspace{20mm}
\vskip0pt plus1filll
20<>. április
\end{center}
\makeatother
%%%%%%%%%%%%%%%%%%%%%%%%%%%%%%%%%%%%%%%%%%%%%%%%%%%%%%%%%%%%%%%%%%%%%%%%%%%%%%%%%%%%%%%%%%%%%%%%%%%%%%%%%%%%%%%%%%%%%%%%%


\newpage
\setcounter{page}{1} \pagenumbering{roman}

%% Ide írd a kivonatot
\begin{abstract}

\noindent Ide jön az absztrakt.

\end{abstract}
\newpage

%% Tartalomjegyzék, ábrák és táblázatok jegyzéke
\microtypesetup{protrusion=false} % oldalszámok szebb illeszkedéséhez
\tableofcontents
\newpage
\listoftables
\newpage
\listoffigures
\microtypesetup{protrusion=true}
\newpage

% Most hogy szépen megvan minden, már csak írni kell
% A \section{} új fejezetet definiál, a \subsection{} és a \subsubsection{} alfejezeteket
% A \label{xy} nagyon hasznos, a szövegben \ref{xy} módon hivatkozhatsz az elem számára (1. fejezet pl),
% a \pageref{xy} az elem oldalszámát adja vissza.
%%%%%%%%%%%%%%%%%%%%%%%%%%%%%%%%%%%%%%%%%%%%%%%%%%%%%%%%%%%%%%%%%%%%%%%%%%%%%%%%%%%%%%%%%%%%%%%%%%%%%%%%%%%%%%%%%%%%%%%%%

\section{Bevezetés} \label{sec:bev}
\pagenumbering{arabic}
%A dolgozat bevezetőjében mindenképpen szerepelnie kell egy konkrét
%kérdésnek, amire a hallgató a dolgozatban választ keres, illetve a témakör megjelölésének,
%amellyel kapcsolatos szakirodalmi áttekintést készít. A tartalmi kifejtés során fontos a logikus
%felépítés, a szakkifejezések megfelelő használata, ha szükséges azok tisztázása. Az összegező
%befejezésnek tartalmaznia kell egy konklúziót a bevezető kérdéssel kapcsolatban. 

\section{Második fejezet}












%%%%%%%%%%%%%%%%%%%%%%%%%%%%%%%%%%%%%%%%%%%%%%%%%%%%%%%%%%%%%%%%%%%%%%%%%%%%%%%%%%%%%%%%%%%%%%%%%%%%%%%%%%%%%%%%%%%%%%%%%

%% Ez helyezi el a hivatkozásjegyzéket
% Zotero-t használóknak: bibtex export segítségével lehet építgetni a saját .bib fájlunkat
\newpage
\bibliographystyle{apacite}
\bibliography{szakdolg}     % ez a .tex fájllal egy mappában lévő szakdolg.bib nevű fájlt használja fel 
                            % a hivatkozásjegyzék item-jeinek forrásaként

% Hivatkozási parancsok a .bib fájlban szereplő elnevezések alapján, és azok megjelenési formája a szövegben:
% \cite{pete_2013} : (Pete, 2013)         \cite[3.~o.]{pete_2013} : (Pete, 2013, 3. o.)
% \citeA{pete_2013} : Pete (2013)         \citeA[3.~o.]{pete_2013} : Pete (2013, 3. o.)
% \citeyear{pete_2013} : (2013)
% és még mások, amiket nem ismerek...

%% Függelék: innentől kezdve a chapterek függelékként vannak jelölve a szövegben (A. függelék, B. függelék ...)
\newpage
\appendix
\section*{Függelék}

\section{Kiegészítő táblázatok}





\end{document}
